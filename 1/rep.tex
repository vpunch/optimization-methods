\documentclass[a4paper,12pt]{report}

\usepackage[utf8]{inputenc}
\usepackage[english,russian]{babel}

\usepackage{amsmath} %математика

\usepackage{tikz,pgfplots} %графики
\pgfplotsset{compat=newest}

\usepackage[left=3cm,right=2cm,top=2cm,bottom=2cm]{geometry} %формат страницы

\usepackage{minted} %листинг


\newcommand{\imgh}[2]{
    \begin{figure}[ht]
        \centering
        \includegraphics[width=1\textwidth]{#1}
        \caption{#2}\label{fig:#1}
    \end{figure}
}

\newcommand{\roots}[4]{
    \begin{tikzpicture}
        \begin{axis}[axis x line=center, axis y line=none, xmin=#1, xmax=#2, xlabel={$x$}]
            \addplot[only marks] coordinates {#3};
            \draw \foreach \p/\s in {#4}{(\p,0.15) node {\s}};
        \end{axis}
    \end{tikzpicture}
}

\newcommand{\eqcom}[1]{
    \left|
    \begin{array}{l} 
        #1
    \end{array}
    \right.
}

\newcommand{\insertcode}[2]{
    \inputminted[
        frame=lines,
        fontsize=\footnotesize,
        linenos
    ]{#2}{#1}
}
 

\begin{document}

\newcommand{\labn}{1}
\begin{titlepage}
    \begin{center}
        Министерство науки и высшего образования\\
        Федеральное государтсвенное бюджетное образовательное учреждение высшего образования\\
        Югорский государственный университет
        \vfill
        \textsc{Отчет о лабораторной работе №\labn}\\
        по дисциплине «Методы оптимизации»
        \vfill
    \end{center}

    \vfill

    \newlength{\datelen}
    \settowidth{\datelen}{«\underline{\hspace{0.7cm}}» \underline{\hspace{2cm}}}
    \hfill\begin{minipage}{0.4\textwidth}
        Выполнил\\

        Студент группы 1162б\\
        \underline{\hspace{\datelen}} Панчишин~И. Р.\\
        «\underline{\hspace{0.7cm}}» \underline{\hspace{2cm}} 2019 г.
    \end{minipage}

    \bigskip

    \hfill\begin{minipage}{0.4\textwidth}
        Принял\\

        Доцент ИЦЭ\\
        \underline{\hspace{\datelen}} Самарин~В. А.\\
        «\underline{\hspace{0.7cm}}» \underline{\hspace{2cm}} 2019 г.
    \end{minipage}

    \vfill

    \begin{center}
        Ханты-Мансийск, 2019
    \end{center}
\end{titlepage}


\subsection*{Цель}

Освоить аналитический метод поиска минимума функции одной переменной.

\subsection*{Задачи}

\begin{enumerate}
    \item Найти наибольшее и наименьшее занчения заданных функций на отрезке.
    \item Исследовать заданные функции на выпуклость, найти экстремумы и точки перегиба.
\end{enumerate}

\subsection*{Ход работы}

Задание 1:
\begin{enumerate}
	\item $\displaystyle y = x + \sqrt{x}, x \in [0; 4]$
	\begin{flalign*}
		&y' = 1 + \frac{1}{2\sqrt{x}} = \frac{2\sqrt{x} + 1}{2\sqrt{x}} = 0&\\
		&\begin{cases}
			x \neq 0\\
			2\sqrt{x} + 1 = 0 \implies \sqrt{x} = -\frac{1}{2}
		\end{cases}
	\end{flalign*}
    \roots{-1}{4}{(0,0)}{2/+}\\
	Ответ: $min = 0$, $max = 4$

	\item $\displaystyle y = e^{2x} - e^{-2x}, x \in [-2; 1]$
	\begin{flalign*}
		&y' = 2e^{2x} + 2e^{-2x} = 2(e^{2x} + 2e^{-2x}) = 0&\\
		&e^{2x} + e^{-2x} = 0&\\
		&\frac{e^{2x + 2x} + 1}{e^{2x}} = 0&\\
		&\begin{cases}
			e^{2x} \neq 0 \implies x \not\to -\infty\\
			e^{4x} = -1
		\end{cases}
	\end{flalign*}
	Ответ: $min = -2$, $max = 1$

	\item $\displaystyle y = \sqrt{4 - x^2}, x \in [-2; 2]$
	\begin{flalign*}
		&y' = \frac{1}{2}(4 - x^2)^{-1/2}(-2x) = -\frac{x}{\sqrt{4 - x^2}} = 0&\\
		&\begin{cases}
			x = 0\\
			4 - x^2 \neq 0 \implies x \neq \pm2
		\end{cases}
	\end{flalign*}
    \roots{-3}{3}{(-2,0) (0,0) (2,0)}{-1/+, 1/-}\\
	Ответ: $min = -2 \cup 2$, $max = 0$

	\item $\displaystyle y = tgx - x, x \in [-\frac{\pi}{4}; \frac{\pi}{4}]$
	\begin{flalign*}
		&y' = \frac{1}{cos^2x} - 1 = 0&\\
		&\frac{1 - cos^2x}{cos^2x} = 0&\\
		&\begin{cases}
			cosx \neq 0\\
			(1 - cosx)(1 + cosx) = 0
		\end{cases} \implies
		\begin{cases}
			x \neq \frac{\pi}{2} + \pi k, k \in Z\\
			x = 2\pi k, k \in Z\\
			x = \pi + 2\pi k, k \in Z
		\end{cases}
	\end{flalign*}
    \roots{-4}{4}{(-pi,0) (-pi/2,0) (0,0) (pi/2,0) (pi,0)}{-3.7/+, -2.35/+, -0.8/+, 0.8/+, 2.35/+, 3.7/+}\\
	Ответ: $min = -\frac{\pi}{4}$, $max = \frac{\pi}{4}$

	\item $\displaystyle y = cos2x + 2x, x \in [-\frac{\pi}{2}; \frac{\pi}{2}]$
	\begin{flalign*}
		&y' = 2(-sin2x) + 2&\\
		&1 - sin2x = 0&\\
		&sin2x = 1&\\
		&2x = \frac{\pi}{2} + 2\pi k, k \in Z \implies x = \frac{\pi}{4} + \pi k&
	\end{flalign*}
	\begin{tikzpicture}
		\coordinate (center) at (0,0);
		\def \r{1}

		\draw (center) circle[radius=\r];
		\fill (center) (180/4:\r) circle[radius=2pt];
		\fill (center) (180/4 + 180:\r) circle[radius=2pt];
	\end{tikzpicture}\\
	Ответ: $min = -\frac{\pi}{2}$, $max = \frac{\pi}{2}$
\end{enumerate}

Задание 2:
\begin{enumerate}
	\item $\displaystyle y = ln(1 + x^3)$	
	\begin{flalign*}
		&1 + x^3 > 0 \implies x > -1&\\
		&y' = \frac{3x^2}{1 + x^3} = 0&\\
		&\begin{cases}
			x = 0\\
			1 + x^3 \neq 0 \implies x \neq -1
		\end{cases}
	\end{flalign*}
    \roots{-2}{2}{(-1,0) (0,0)}{-1.5/-, -0.5/+, 1/+}
	\begin{flalign*}
		&y'' = \frac{6x(1 + x^3) - 3x^2 3x^2}{(1 + x^3)^2} = \frac{3x(2 - x^3)}{(1 + x^3)^2}&\\
		&\begin{cases}
			1 + x^3 \neq 0 \implies x \neq -1\\
			x = 0\\
			x = \sqrt[3]{2}
		\end{cases}
	\end{flalign*}
    \roots{-2}{2}{(-1,0) (0,0) (2^(1/3),0)}{-1.5/-, -0.5/-, 1/+, 2/-}\\

	\begin{tikzpicture}
		\begin{axis}[
			axis x line=center,
			axis y line=center,
			xmin=-2, xmax=2,
			xlabel={$x$},
			ylabel={$y$},
			samples at={-2,-1.9,...,2}
		]
			\addplot[smooth]{ln(1 + x^3)};
		\end{axis}
	\end{tikzpicture}
\end{enumerate}

Дополнительное задание:
\begin{enumerate}
	\item $\displaystyle f(x) = x^3 - 27x + 5 \to min, x \in [-4; 4]$
	\begin{flalign*}
		&f'(x) = 3x^2 - 27 = 0 \implies x = \pm3&
	\end{flalign*}
    \roots{-4}{4}{(-3,0) (3,0)}{-4/+, 0/-, 4/+}\\
    Ответ: $f(3) = 27 - 81 + 5 = -49$

	\item $\displaystyle f(x) = x^2 - 3x + xlnx$
	\begin{flalign*}
		&f'(x) = 2x - 3 + lnx + 1 = 2x - 2 + lnx = 0&\\
		&lne^{2x} - lne^{2} + lnx = 0&\\
		&ln(\frac{e^{2x}x}{e^{2}}) = 0&\\
		\intertext{По свойству степеней: $a^{0} = 1$}
		&\frac{e^{2x}x}{e^{2}} = 1 \eqcom{2x = t\\x = \frac{t}{2}}&\\
		&\frac{te^{t}}{2e^{2}} = 1 \implies te^{t} = 2e^{2}&\\
		\intertext{По определению W-функции Ламберта: $x = W(xe^{x})$}
		&t = W(te^{t}) = W(2e^{2}) = 2&\\
		&2x = 2 \implies x = 1&
	\end{flalign*}
    \roots{0}{2}{(1,0)}{}\\
	Так как каждый локальный минимум является глобальным, ф-я унимодальна.
\end{enumerate}

\subsection*{Вывод}

Выполнил все задания и освоил аналитический метод поиска минимума функции одной переменной.

\end{document}
