% Copyright © 2019 Panchishin Ivan

\documentclass[a4paper,12pt]{report}

\usepackage[utf8]{inputenc}
\usepackage[english,russian]{babel}

\usepackage{amsmath} %математика

\usepackage{tikz,pgfplots} %графики
\pgfplotsset{compat=newest}

\usepackage[left=3cm,right=2cm,top=2cm,bottom=2cm]{geometry} %формат страницы

\usepackage{minted} %листинг


\newcommand{\imgh}[2]{
    \begin{figure}[ht]
        \centering
        \includegraphics[width=1\textwidth]{#1}
        \caption{#2}\label{fig:#1}
    \end{figure}
}

\newcommand{\roots}[4]{
    \begin{tikzpicture}
        \begin{axis}[axis x line=center, axis y line=none, xmin=#1, xmax=#2, xlabel={$x$}]
            \addplot[only marks] coordinates {#3};
            \draw \foreach \p/\s in {#4}{(\p,0.15) node {\s}};
        \end{axis}
    \end{tikzpicture}
}

\newcommand{\eqcom}[1]{
    \left|
    \begin{array}{l} 
        #1
    \end{array}
    \right.
}

\newcommand{\insertcode}[2]{
    \inputminted[
        frame=lines,
        fontsize=\footnotesize,
        linenos
    ]{#2}{#1}
}


\begin{document}
    
\newcommand{\labn}{10}
\begin{titlepage}
    \begin{center}
        Министерство науки и высшего образования\\
        Федеральное государтсвенное бюджетное образовательное учреждение высшего образования\\
        Югорский государственный университет
        \vfill
        \textsc{Отчет о лабораторной работе №\labn}\\
        по дисциплине «Методы оптимизации»
        \vfill
    \end{center}

    \vfill

    \newlength{\datelen}
    \settowidth{\datelen}{«\underline{\hspace{0.7cm}}» \underline{\hspace{2cm}}}
    \hfill\begin{minipage}{0.4\textwidth}
        Выполнил\\

        Студент группы 1162б\\
        \underline{\hspace{\datelen}} Панчишин~И. Р.\\
        «\underline{\hspace{0.7cm}}» \underline{\hspace{2cm}} 2019 г.
    \end{minipage}

    \bigskip

    \hfill\begin{minipage}{0.4\textwidth}
        Принял\\

        Доцент ИЦЭ\\
        \underline{\hspace{\datelen}} Самарин~В. А.\\
        «\underline{\hspace{0.7cm}}» \underline{\hspace{2cm}} 2019 г.
    \end{minipage}

    \vfill

    \begin{center}
        Ханты-Мансийск, 2019
    \end{center}
\end{titlepage}


\subsection*{Цель}

Изучить численные методы поиска условного экстремума.

\subsection*{Задачи}

\begin{enumerate}
    \item Ознакомиться с методами штрафных функций.
    \item Написать программную реализацию методов и применить ее для нахождения минимума.
\end{enumerate}

\subsection*{Ход работы}

Исходный код на языке \textit{Octave}, используемый для иллюстрации поиска минимума, приведен в листинге ниже. Там можно втретить функции, которые были реализованы и рассмотренны в предыдущих лабораторных работах.
\insertcode{code.m}{octave}

Исходный код штрафных функций:
\insertcode{../code/penalty.m}{octave}

Результат поиска минимума приведен на рисунке~\ref{fig:f}.
Можно заметить, что в ограничивающей области линии уровней или значения целевой и штрафной функций совпадают, что объясняется тем, что штраф не накладывается и штрафная функция вырождается в целевую.
Также можно заметить, что штрафная функция имеет свой минимум, смещенный в зависимости от велечины масштабного коэффициента штрафа в строну ограничивающей области.
\imgh{f}{Применение методов штрафных функций}

\subsection*{Вывод}

Ознакомился с методами штрафных функций, реализовал и применил их.
В процессе выполнения столкнулся с проблемой "преодоления барьера" в методе внутренней точки. Так как штраф накладывается по мере приближения к границе области, важно не перепрыгнуть ее. Для этого нужно правильно подобрать отрезок поиска или шаг.

\end{document}
