\documentclass[a4paper,12pt]{report}

\usepackage[utf8]{inputenc}
\usepackage[english,russian]{babel}

\usepackage{amsmath} %математика

\usepackage{tikz,pgfplots} %графики
\pgfplotsset{compat=newest}

\usepackage[left=3cm,right=2cm,top=2cm,bottom=2cm]{geometry} %формат страницы

\usepackage{minted} %листинг


\newcommand{\imgh}[2]{
    \begin{figure}[ht]
        \centering
        \includegraphics[width=1\textwidth]{#1}
        \caption{#2}\label{fig:#1}
    \end{figure}
}

\newcommand{\roots}[4]{
    \begin{tikzpicture}
        \begin{axis}[axis x line=center, axis y line=none, xmin=#1, xmax=#2, xlabel={$x$}]
            \addplot[only marks] coordinates {#3};
            \draw \foreach \p/\s in {#4}{(\p,0.15) node {\s}};
        \end{axis}
    \end{tikzpicture}
}

\newcommand{\eqcom}[1]{
    \left|
    \begin{array}{l} 
        #1
    \end{array}
    \right.
}

\newcommand{\insertcode}[2]{
    \inputminted[
        frame=lines,
        fontsize=\footnotesize,
        linenos
    ]{#2}{#1}
}


\begin{document} 

\newcommand{\labn}{2}
\begin{titlepage}
    \begin{center}
        Министерство науки и высшего образования\\
        Федеральное государтсвенное бюджетное образовательное учреждение высшего образования\\
        Югорский государственный университет
        \vfill
        \textsc{Отчет о лабораторной работе №\labn}\\
        по дисциплине «Методы оптимизации»
        \vfill
    \end{center}

    \vfill

    \newlength{\datelen}
    \settowidth{\datelen}{«\underline{\hspace{0.7cm}}» \underline{\hspace{2cm}}}
    \hfill\begin{minipage}{0.4\textwidth}
        Выполнил\\

        Студент группы 1162б\\
        \underline{\hspace{\datelen}} Панчишин~И. Р.\\
        «\underline{\hspace{0.7cm}}» \underline{\hspace{2cm}} 2019 г.
    \end{minipage}

    \bigskip

    \hfill\begin{minipage}{0.4\textwidth}
        Принял\\

        Доцент ИЦЭ\\
        \underline{\hspace{\datelen}} Самарин~В. А.\\
        «\underline{\hspace{0.7cm}}» \underline{\hspace{2cm}} 2019 г.
    \end{minipage}

    \vfill

    \begin{center}
        Ханты-Мансийск, 2019
    \end{center}
\end{titlepage}


\subsection*{Цель}

Изучить прямые методы минимизации.

\subsection*{Задачи}

\begin{enumerate}
	\item Реализовать следующие три метода минимизации: дихотомии, золотого сечения и Фибоначчи.
	\item Изучить зависимость числа вычислений функции (скорости работы) от заданной точности.
\end{enumerate}

\subsection*{Ход работы}

Реализовал требуемые методы на языке программирования Octave (свободная реализация Matlab). 
Исходный код, представленный ниже, позволяет определить минимум функции (унимодальной), 
визуализировать изменение отрезка поиска на каждой итерации, 
а также построить зависимость скорости поиска от заданной точности.
\insertcode{code.m}{octave}

Результат нахождения минимума представлен на Рис.~\ref{fig:f}. Он близок к результату работы встроенной функции \textit{fminbnd}.
\imgh{f}{Минимум функции}

Зависимость числа вычислений от точности представлена на Рис.~\ref{fig:e}. Наблюдается экспоненциальный рост количества вычислений с увеличением точности.
Метод золотого сечения показал себя лучше остальных, рассматриваемых, методов.
\imgh{e}{Зависимость скорости от точности}

\subsection*{Вывод}

Реализовал прямые методы минимизации: дихотомии, золотого сечения и Фибоначчи. Сравнил их работу.

\end{document}
