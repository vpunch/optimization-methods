\documentclass[a4paper,12pt]{report}

\usepackage[utf8]{inputenc}
\usepackage[english,russian]{babel}

\usepackage{amsmath} %математика

\usepackage{tikz,pgfplots} %графики
\pgfplotsset{compat=newest}

\usepackage[left=3cm,right=2cm,top=2cm,bottom=2cm]{geometry} %формат страницы

\usepackage{minted} %листинг


\newcommand{\imgh}[2]{
    \begin{figure}[ht]
        \centering
        \includegraphics[width=1\textwidth]{#1}
        \caption{#2}\label{fig:#1}
    \end{figure}
}

\newcommand{\roots}[4]{
    \begin{tikzpicture}
        \begin{axis}[axis x line=center, axis y line=none, xmin=#1, xmax=#2, xlabel={$x$}]
            \addplot[only marks] coordinates {#3};
            \draw \foreach \p/\s in {#4}{(\p,0.15) node {\s}};
        \end{axis}
    \end{tikzpicture}
}

\newcommand{\eqcom}[1]{
    \left|
    \begin{array}{l} 
        #1
    \end{array}
    \right.
}

\newcommand{\insertcode}[2]{
    \inputminted[
        frame=lines,
        fontsize=\footnotesize,
        linenos
    ]{#2}{#1}
}


\begin{document}

\newcommand{\labn}{3}
\begin{titlepage}
    \begin{center}
        Министерство науки и высшего образования\\
        Федеральное государтсвенное бюджетное образовательное учреждение высшего образования\\
        Югорский государственный университет
        \vfill
        \textsc{Отчет о лабораторной работе №\labn}\\
        по дисциплине «Методы оптимизации»
        \vfill
    \end{center}

    \vfill

    \newlength{\datelen}
    \settowidth{\datelen}{«\underline{\hspace{0.7cm}}» \underline{\hspace{2cm}}}
    \hfill\begin{minipage}{0.4\textwidth}
        Выполнил\\

        Студент группы 1162б\\
        \underline{\hspace{\datelen}} Панчишин~И. Р.\\
        «\underline{\hspace{0.7cm}}» \underline{\hspace{2cm}} 2019 г.
    \end{minipage}

    \bigskip

    \hfill\begin{minipage}{0.4\textwidth}
        Принял\\

        Доцент ИЦЭ\\
        \underline{\hspace{\datelen}} Самарин~В. А.\\
        «\underline{\hspace{0.7cm}}» \underline{\hspace{2cm}} 2019 г.
    \end{minipage}

    \vfill

    \begin{center}
        Ханты-Мансийск, 2019
    \end{center}
\end{titlepage}


\subsection*{Цель}

Изучить прямые методы минимизации.

\subsection*{Задачи}

\begin{enumerate}
    \item Реализовать метод парабол (полиномиальной интерполяции).
\end{enumerate}

\subsection*{Ход работы}

Реализовал метод парабол на языке Octave. Кроме самого метода 
в листинге содержится алгоритм грубой локализации минимума, который используется для подбора выпуклой тройки точек, лежащих на 
уменьшенном отрезке поиска. Кроме грубой локализации можно использовать, например, метод золтого сечения или просто выбрать случайную точку, если
отрезок поиска небольшой (рассматриваются унимодальные функции).
\insertcode{code.m}{octave}

Результат работы метода представлен на Рис.~\ref{fig:f}.
Здесь изображена исходная функция со своим минимумом и аппроксимирующие параболы.
\imgh{f}{Минимум функции}

\subsection*{Вывод}

Реализовал метод парабол, поставленную задачу выполнил.

\end{document}
