\documentclass[a4paper,12pt]{report}

\usepackage[utf8]{inputenc}
\usepackage[english,russian]{babel}

\usepackage{amsmath} %математика

\usepackage{tikz,pgfplots} %графики
\pgfplotsset{compat=newest}

\usepackage[left=3cm,right=2cm,top=2cm,bottom=2cm]{geometry} %формат страницы

\usepackage{minted} %листинг


\newcommand{\imgh}[2]{
    \begin{figure}[ht]
        \centering
        \includegraphics[width=1\textwidth]{#1}
        \caption{#2}\label{fig:#1}
    \end{figure}
}

\newcommand{\roots}[4]{
    \begin{tikzpicture}
        \begin{axis}[axis x line=center, axis y line=none, xmin=#1, xmax=#2, xlabel={$x$}]
            \addplot[only marks] coordinates {#3};
            \draw \foreach \p/\s in {#4}{(\p,0.15) node {\s}};
        \end{axis}
    \end{tikzpicture}
}

\newcommand{\eqcom}[1]{
    \left|
    \begin{array}{l} 
        #1
    \end{array}
    \right.
}

\newcommand{\insertcode}[2]{
    \inputminted[
        frame=lines,
        fontsize=\footnotesize,
        linenos
    ]{#2}{#1}
}


\begin{document}
        
\newcommand{\labn}{5}
\begin{titlepage}
    \begin{center}
        Министерство науки и высшего образования\\
        Федеральное государтсвенное бюджетное образовательное учреждение высшего образования\\
        Югорский государственный университет
        \vfill
        \textsc{Отчет о лабораторной работе №\labn}\\
        по дисциплине «Методы оптимизации»
        \vfill
    \end{center}

    \vfill

    \newlength{\datelen}
    \settowidth{\datelen}{«\underline{\hspace{0.7cm}}» \underline{\hspace{2cm}}}
    \hfill\begin{minipage}{0.4\textwidth}
        Выполнил\\

        Студент группы 1162б\\
        \underline{\hspace{\datelen}} Панчишин~И. Р.\\
        «\underline{\hspace{0.7cm}}» \underline{\hspace{2cm}} 2019 г.
    \end{minipage}

    \bigskip

    \hfill\begin{minipage}{0.4\textwidth}
        Принял\\

        Доцент ИЦЭ\\
        \underline{\hspace{\datelen}} Самарин~В. А.\\
        «\underline{\hspace{0.7cm}}» \underline{\hspace{2cm}} 2019 г.
    \end{minipage}

    \vfill

    \begin{center}
        Ханты-Мансийск, 2019
    \end{center}
\end{titlepage}


\subsection*{Цель}

Научиться находить безусловные экстремумы функций нескольких переменных.

\subsection*{Задачи}

\begin{enumerate}
    \item Исследовать на максимум и минимум заданные функции.
    \item Найти точки безусловного экстремума функции согласно варианту.
\end{enumerate}

\subsection*{Ход работы}

Исследуем на максимум и минимум следующие функции:
\begin{enumerate}
    \item $\displaystyle f(x) = f(x_1, x_2) = -x_1^{2} - x_1x_2 - x_2^{2} + x_1 + x_2$
    \begin{flalign*}
        \intertext{Поиск стационарных точек и проверка необходимого условия экстремума первого порядка (порядок условий определяется порядком используемых производных):}
        &\nabla(f) = (\frac{\partial f}{\partial x_1}, \frac{\partial f}{\partial x_2}) = (-2x_1 - x_2 + 1, -x_1 - 2x_2 + 1) = 0&\\
        &\begin{cases}
            -2x_1 - x_2 + 1 = 0\\
            -x_1 - 2x_2 + 1 = 0
        \end{cases} \implies
        \begin{cases}
            -2 + 4x_2 - x_2 + 1 = 0\\
            x_1 = 1 - 2x_2
        \end{cases} \implies
        \begin{cases}
            x_2 = \frac{1}{3}\\
            x_1 = \frac{1}{3}
        \end{cases}
        \intertext{Стационарная точка одна --- $x^{*} = (\frac{1}{3}, \frac{1}{3})$.\\Рассмотрим матрицу Гессе и проверим необходимое условие экстремума второго порядка:}
        &H(f) =
        \begin{bmatrix}
            \frac{\partial f}{\partial x_1x_1} & \frac{\partial f}{\partial x_1x_2}\\
            \frac{\partial f}{\partial x_2x_1} & \frac{\partial f}{\partial x_2x_2}
        \end{bmatrix} \implies
        \begin{bmatrix}
            -2 & -1\\
            -1 & -2
        \end{bmatrix}\\
        &\Delta_1 = -2 < 0&\\
        &\Delta_2 = (-2 \cdot -2) - (-1 \cdot -1) =  3 > 0&
        \intertext{Матрица является отрицательно определенной, т. е. $H(x^{*}) < 0$, поэтому максимумом является}
        &f(x*) = -\frac{1}{3}^{2} - \frac{1}{3} \cdot \frac{1}{3} - \frac{1}{3}^{2} + \frac{1}{3} + \frac{1}{3} = \frac{1}{3}&
    \end{flalign*}
    Ответ: $f(\frac{1}{3}, \frac{1}{3}) = \frac{1}{3}$ --- максимум

    \item $\displaystyle f(x_1, x_2) = x_1^{3} + x_2^{3} - 3x_1x_2$
    \begin{flalign*}
        &\nabla(f) = (3x_1^{2} - 3x_2, 3x_2^{2} - 3x_1) = 0&\\
        &\begin{cases}
            x_1^{2} - x_2 = 0\\
            x_2^{2} - x_1 = 0
        \end{cases} \implies
        \begin{cases}
            x_2 = x_1^{2}\\
            x_1(x_1^{3} - 1) = 0
        \end{cases} \implies
        \begin{cases}
            x_{11} = 0\\
            x_{12} = 1\\
            x_{21} = 0\\
            x_{22} = 1
        \end{cases}\\
        &H(f) =
        \begin{bmatrix}
            6x_1 & -3\\
            -3 & 6x_2
        \end{bmatrix}\\
        &H(x_1^{*}) =
        \begin{bmatrix}
            0 & -3\\
            -3 & 0
        \end{bmatrix}
        \Delta_1 = 0, \Delta_2 = -9\\
        &H(x_2^{*}) =
        \begin{bmatrix}
            6 & -3\\
            -3 & 6
        \end{bmatrix}
        \Delta_1 = 6, \Delta_2 = 27
    \end{flalign*}
    Ответ: $f(1, 1) = -1$ --- минимум

    \item $\displaystyle f(x_1, x_2) = e^{2x_1}(x_1 + x_2^{2} + 2x_2)$
    \begin{flalign*}
        &\nabla(f) = (2e^{2x_1}(x_1 + x_2^{2} + 2x_2) + e^{2x_1}, e^{2x_1}(2x_2 + 2)) = 0&\\
        &\begin{cases}
            e^{2x_1}(2(x_1 + x_2^{2} + 2x_2) + 1) = 0\\
            e^{2x_1}(2x_2 + 2) = 0 
        \end{cases} \implies
        \begin{cases}
            e^{2x_1} = 0\\
            x_1 + x_2^{2} + 2x_2 = -\frac{1}{2}\\
            x_2 + 1 = 0
        \end{cases} \implies
        \begin{cases}
            x_1 = \frac{1}{2}\\
            x_2 = -1
        \end{cases}\\
        &H_{11} = 2e^{2x_1}(2(x_1 + x_2^{2} + 2x_2) + 1) + 2e^{2x_1} = 4e^{2x_1}(x_1 + x_2^{2} + 2x_2 + 1)&\\
        &H_{12} = 2e^{2x_1}(2x_2 + 2) = 4e^{2x_1}(x_2 + 1)&\\
        &H_{22} = 2e^{2x_1}&\\
        &H(f) = 
        \begin{bmatrix}
            4e^{2x_1}(x_1 + x_2^{2} + 2x_2 + 1) & 4e^{2x_1}(x_2 + 1)\\
            H_{12} & 2e^{2x_1}
        \end{bmatrix}\\
        &H(x^{*}) =
        \begin{bmatrix}
            2e & 0\\
            0 & 2e
        \end{bmatrix}
        \Delta_1 = 2e, \Delta_2 = 4e^{2}
    \end{flalign*}
    Ответ: $f(\frac{1}{2}, -1) = -\frac{e}{2}$ --- минимум

    \item $\displaystyle f(x_1, x_2) = x_1^{2} - x_1x_2 + x_2^{2}$
    \begin{flalign*}
        &\nabla(f)  = (2x_1 - x_2, -x_1 + 2x_2) = 0&\\
        &\begin{cases}
            2x_1 - x_2 = 0\\
            -x_1 + 2x_2 = 0
        \end{cases} \implies
        \begin{cases}
            x_2 = 2x_1\\
            -x_1 + 4x_1 = 0
        \end{cases} \implies
        \begin{cases}
            x_1 = 0\\
            x_2 = 0
        \end{cases}\\
        &H(f) =
        \begin{bmatrix}
            2 & -1\\
            -1 & 2
        \end{bmatrix}
        \Delta_1 = 2, \Delta_2 = 3
    \end{flalign*}
    Ответ: $f(0, 0) = 0$ --- минимум

    \item $\displaystyle f(x_1, x_2) = x_1^{2} - 2x_1x_2 + 2x_2^{2} + 2x_1$
    \begin{flalign*}
        &\nabla(f) = 0 \implies
        \begin{cases}
            2x_1 - 2x_2 + 2 = 0\\
            -2x_1 + 4x_2 = 0
        \end{cases} \implies
        \begin{cases}
            x_1 - x_2 + 1 = 0\\
            -x_1 + 2x_2 = 0
        \end{cases} \implies
        \begin{cases}
            x_1 = -2\\
            x_2 = -1
        \end{cases}\\
        &H(f) =
        \begin{bmatrix}
            1 & -1\\
            -1 & 2
        \end{bmatrix}
        \Delta_1 = 1, \Delta_2 = 1
    \end{flalign*}
    Ответ: $f(-2 , -1) = -2$ --- минимум

    \item $\displaystyle f(x_1, x_2) = 2 - \sqrt[3]{x_1^{2} + x_2^{2}}$
    \begin{flalign*}
        &\nabla(f) = (-\frac{2x_1}{3 \sqrt[3]{(x_1^{2} + x_2^{2})^{2}}}, -\frac{2x_2}{3 \sqrt[3]{(x_1^{2} + x_2^{2})^{2}}}) = 0&\\
        &\begin{cases}
            x_1 = 0\\
            x_2 = 0
        \end{cases}\\
        &H_{11} = - \frac{2 \cdot 3(x_1^{2} + x_2^{2})^{\frac{2}{3}} - 2x_1 \cdot 3 \cdot \frac{2}{3}(x_1^{2} + x_2^{2})^{-\frac{1}{3}} 2x_1}{9(x_1^{2} + x_2^{2})^{\frac{4}{3}}} =&\\
        &= \frac{8}{9}x_1^{2}(x_1^{2} + x_2^{2})^{-\frac{1}{3} - \frac{4}{3}} - \frac{2}{3}(x_1^{2} + x_2^{2})^{\frac{2}{3} - \frac{4}{3}} = &\\
        &= \frac{8x_1^{2}}{9(x_1^{2} + x_2^{2})^{\frac{5}{3}}} - \frac{2}{3(x_1^{2} + x_2^{2})^{\frac{2}{3}}}&\\
        &H_{12} = -\frac{2x_1}{3}(-\frac{2}{3})(x_1^{2} + x_2^{2})^{-\frac{5}{3}}2x_2 = \frac{8x_2x_1}{9(x_1^{2} + x_2^{2})^{5/3}}&\\
        &H_{22} = \frac{8x_2^{2}}{9(x_1^{2} + x_2^{2})^{5/3}} - \frac{2}{3(x_1^{2} + x_2^{2})^{2/3}}&
    \end{flalign*}

    \item $\displaystyle f(x_1, x_2) = x_1^{3} - 2x_2^{3} - 3x_1 + 6x_2$
    \begin{flalign*}
        &\nabla(f) = 0 \implies
        \begin{cases}
            3x_1^{2} - 3 = 0\\
            -6x_2^{2} + 6 = 0
        \end{cases} \implies
        \begin{cases}
            x_1^{2} - 1 = 0\\
            -x_2^{2} + 1 = 0
        \end{cases} \implies
        \begin{cases}
            x_{1} = \pm 1\\
            x_{2} = \pm 1
        \end{cases}&\\
        &H(f) =
        \begin{bmatrix}
            6x_1 & 0\\
            0 & -12x_2
        \end{bmatrix}\\
        &H(1, 1) =
        \begin{bmatrix}
            6 & 0\\
            0 & -12
        \end{bmatrix}
        \Delta_1 = 6, \Delta_2 = -72\\
        &H(-1, 1) =
        \begin{bmatrix}
            -6 & 0\\
            0 & -12
        \end{bmatrix}
        \Delta_1 = -6, \Delta_2 = 72\\
        &H(1, -1) =
        \begin{bmatrix}
            6 & 0\\
            0 & 12
        \end{bmatrix}
        \Delta_1 = 6, \Delta_2 = 72\\
        &H(-1, -1) =
        \begin{bmatrix}
            -6 & 0\\
            0 & 12
        \end{bmatrix}
        \Delta_1 = -6, \Delta_2 = -72
    \end{flalign*}
    Ответ: $f(1, -1) = -6$ --- минимум, $f(-1, 1) = 6$ --- максимум

    \item $f(x_1, x_2) = e^{-(x_1^{2} + x_2^{2})}(2x_1^{2} + x_2^{2})$
    \begin{flalign*}
        &\nabla_1(f) = e^{-(x_1^{2} + x_2^{2})}(-2x_1)(2x_1^{2} + x_2^{2}) + e^{-(x_1^{2} + x_2^{2})}4x_1 = 2x_1e^{-x_2^{2}}(-2x_1^{2} - x_2^{2} + 2)&
        &&
    \end{flalign*}
\end{enumerate}

Рассмотрим функцию $\displaystyle 3x_1^{2} - 3x_1x_2 + 1x_2^{2} - 7x_1 - 7x_2$:
\begin{flalign*}
    &\nabla = (6x_1 - 3x_2 + 7, -3x_1 + 2x_2 - 7 = 0)&
    \intertext{Стационарные точки:}
    &\begin{cases}
        6x_1 - 3x_2 + 7 = 0\\
        -3x_1 + 2x_2 - 7 = 0 \big| \cdot 2, 1 + 2
    \end{cases} \implies
    \begin{cases}
        x_2 - 7 = 0\\
        x_1 = \frac{2x_2 - 7}{3}
    \end{cases} \implies
    \begin{cases}
        x_2 = 7\\
        x_1 = \frac{7}{3} \approx 2.33
    \end{cases}
\end{flalign*}
Программный код для расчета минимума функции:
\insertcode{code.m}{octave}
Чтобы ускорить поиск минимума, можно определить градиет рассматриваемой функции, иначе градиент будет будет вычисляться с помощью конечных разностей.
Результат работы представлен на Рис.~\ref{fig:f}.
\imgh{f}{Минимум функции}
Контур рассматриваемой функции представлен на Рис.~\ref{fig:con}.
\imgh{con}{Контур функции}

\subsection*{Вывод}

Научился находить безусловные экстремумы функций нескольких переменных, выполнил поставленные задачи аналитически и программно.

\end{document}
