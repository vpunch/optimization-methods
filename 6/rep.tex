\documentclass[a4paper,12pt]{report}

\usepackage[utf8]{inputenc}
\usepackage[english,russian]{babel}

\usepackage{amsmath} %математика

\usepackage{tikz,pgfplots} %графики
\pgfplotsset{compat=newest}

\usepackage[left=3cm,right=2cm,top=2cm,bottom=2cm]{geometry} %формат страницы

\usepackage{minted} %листинг


\newcommand{\imgh}[2]{
    \begin{figure}[ht]
        \centering
        \includegraphics[width=1\textwidth]{#1}
        \caption{#2}\label{fig:#1}
    \end{figure}
}

\newcommand{\roots}[4]{
    \begin{tikzpicture}
        \begin{axis}[axis x line=center, axis y line=none, xmin=#1, xmax=#2, xlabel={$x$}]
            \addplot[only marks] coordinates {#3};
            \draw \foreach \p/\s in {#4}{(\p,0.15) node {\s}};
        \end{axis}
    \end{tikzpicture}
}

\newcommand{\eqcom}[1]{
    \left|
    \begin{array}{l} 
        #1
    \end{array}
    \right.
}

\newcommand{\insertcode}[2]{
    \inputminted[
        frame=lines,
        fontsize=\footnotesize,
        linenos
    ]{#2}{#1}
}


\begin{document}
    
\newcommand{\labn}{6}
\begin{titlepage}
    \begin{center}
        Министерство науки и высшего образования\\
        Федеральное государтсвенное бюджетное образовательное учреждение высшего образования\\
        Югорский государственный университет
        \vfill
        \textsc{Отчет о лабораторной работе №\labn}\\
        по дисциплине «Методы оптимизации»
        \vfill
    \end{center}

    \vfill

    \newlength{\datelen}
    \settowidth{\datelen}{«\underline{\hspace{0.7cm}}» \underline{\hspace{2cm}}}
    \hfill\begin{minipage}{0.4\textwidth}
        Выполнил\\

        Студент группы 1162б\\
        \underline{\hspace{\datelen}} Панчишин~И. Р.\\
        «\underline{\hspace{0.7cm}}» \underline{\hspace{2cm}} 2019 г.
    \end{minipage}

    \bigskip

    \hfill\begin{minipage}{0.4\textwidth}
        Принял\\

        Доцент ИЦЭ\\
        \underline{\hspace{\datelen}} Самарин~В. А.\\
        «\underline{\hspace{0.7cm}}» \underline{\hspace{2cm}} 2019 г.
    \end{minipage}

    \vfill

    \begin{center}
        Ханты-Мансийск, 2019
    \end{center}
\end{titlepage}


\subsection*{Цель}

Изучить метод градиентного спуска для задач минимизации.

\subsection*{Задачи}

\begin{enumerate}
    \item Написать программную реализацию рассматриваемого метода.
    \item Найти минимум функции, используя метод градиентного спуска.
    \item Сравнить сходимость при различных значениях шага.
\end{enumerate}

\subsection*{Ход работы}

В листинге ниже представлена программная реализация нескольких типов метода градиентного спуска.
Она также будет использована в следующих лабораторных работах на тему градиентного спуска.
\insertcode{code.m}{octave}
\insertcode{../code/graddesc.m}{octave}

Результат поиска минимума функции из предыдущей лабораторной работы представлен на Рис.~\ref{fig:f}.
\imgh{f}{Минимум функции}

Количество вычислений функции при различном шаге отражено на Рис.~\ref{fig:n}.
Каждый график представлен в виде набора уровней, где указана точка минимума, а также путь передвижения начального приближения.
\imgh{n}{Скорость работы при заданном шаге}

\subsection*{Вывод}

Выполнил все поставленные задачи. 
Наилучшую сходимость обеспечивает наименьший шаг, но наилучшую скорость работы алгоритма обеспечило значение 0.2.

\end{document}
