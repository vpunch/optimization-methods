% Copyright © 2019 Panchishin Ivan

\documentclass[a4paper,12pt]{report}

\usepackage[utf8]{inputenc}
\usepackage[english,russian]{babel}

\usepackage{amsmath} %математика

\usepackage{tikz,pgfplots} %графики
\pgfplotsset{compat=newest}

\usepackage[left=3cm,right=2cm,top=2cm,bottom=2cm]{geometry} %формат страницы

\usepackage{minted} %листинг


\newcommand{\imgh}[2]{
    \begin{figure}[ht]
        \centering
        \includegraphics[width=1\textwidth]{#1}
        \caption{#2}\label{fig:#1}
    \end{figure}
}

\newcommand{\roots}[4]{
    \begin{tikzpicture}
        \begin{axis}[axis x line=center, axis y line=none, xmin=#1, xmax=#2, xlabel={$x$}]
            \addplot[only marks] coordinates {#3};
            \draw \foreach \p/\s in {#4}{(\p,0.15) node {\s}};
        \end{axis}
    \end{tikzpicture}
}

\newcommand{\eqcom}[1]{
    \left|
    \begin{array}{l} 
        #1
    \end{array}
    \right.
}

\newcommand{\insertcode}[2]{
    \inputminted[
        frame=lines,
        fontsize=\footnotesize,
        linenos
    ]{#2}{#1}
}


\begin{document}
    
\newcommand{\labn}{9}
\begin{titlepage}
    \begin{center}
        Министерство науки и высшего образования\\
        Федеральное государтсвенное бюджетное образовательное учреждение высшего образования\\
        Югорский государственный университет
        \vfill
        \textsc{Отчет о лабораторной работе №\labn}\\
        по дисциплине «Методы оптимизации»
        \vfill
    \end{center}

    \vfill

    \newlength{\datelen}
    \settowidth{\datelen}{«\underline{\hspace{0.7cm}}» \underline{\hspace{2cm}}}
    \hfill\begin{minipage}{0.4\textwidth}
        Выполнил\\

        Студент группы 1162б\\
        \underline{\hspace{\datelen}} Панчишин~И. Р.\\
        «\underline{\hspace{0.7cm}}» \underline{\hspace{2cm}} 2019 г.
    \end{minipage}

    \bigskip

    \hfill\begin{minipage}{0.4\textwidth}
        Принял\\

        Доцент ИЦЭ\\
        \underline{\hspace{\datelen}} Самарин~В. А.\\
        «\underline{\hspace{0.7cm}}» \underline{\hspace{2cm}} 2019 г.
    \end{minipage}

    \vfill

    \begin{center}
        Ханты-Мансийск, 2019
    \end{center}
\end{titlepage}


\subsection*{Цель}

Изучить аналитические методы нахожения условного экстремума функции двух переменных.

\subsection*{Задачи}

\begin{enumerate}
    \item Найти условные экстремумы при ограничениях типа равенств заданных функций.
    \item Найти наибольшее и наименьшее значения заданных функций в замкнутой области.
\end{enumerate}

\subsection*{Ход работы}
    \[
    f(x) = f(x_1, x_2) = \frac{1}{x_1} + \frac{1}{x_2},\ x_1 + x_2 = 2
    .\]

    Графическое описание приведено на рисунке~\ref{fig:1task}. 
    Его можно использовать для проверки ответа. 
    Рисунок содержит графцик целевой функции и ее сечение (график функции по точкам на прямой уравнения связи), спроецированное на ось $x_1$, а также без проекции.
    Чтобы избежать проекции сечения на ось $x_1$, координаты $x_1$ разделил на косинус угла наклона прямой.
    \imgh{1task}{Пример 1}

    Решим пример, выразив один из аргументов из уравнения связи:
    \begin{flalign*}
        &x_2 = 2 - x_1 \implies f(x) = \frac{1}{x_1} + \frac{1}{2 - x_1} = \frac{2 - x_1 + x_1}{x_1(2 - x_1)} = \frac{2}{2x_1 - x_1^{2}}&
    \end{flalign*}
    Геометрически мы перешли к поиску экстремума на сечении функции плоскостью, которую образует уравнение связи.
    Сечение проецируется на ось $x_1$ как показано на рисунке~\ref{fig:1x1section}.
    \imgh{1x1section}{Выраженная функция}

    Найдем стационарные точки:
    \begin{flalign*}
        f'(x) &= \frac{-(2 - 2x_1)2}{(2x_1 - x_1^{2})^{2}} = - \frac{4(1 - x_1)}{(2x_1 - x_1^{2})^{2}}&\\
        1 - x_1 &= 0 \implies x_1^{*} = 1&
        \intertext{Проверим достаточное условие наличия экстремума:}
        f''(x) &= -4(\frac{-1(2x_1 - x_1^{2})^{2} - (1 - x_1)(2 - 2x_1)2(2x_1 - x_1^{2})}{(2x_1 - x_1^{2})^{4}})&\\
        &= \frac{4}{(2x_1 - x_1^{2})^{2}} + \frac{4((1 - x_1)(2 - 2x_1)2)}{(2x_1 - x_1^{2})^{3}} = \frac{4}{(2x_1 - x_1^{2})^{2}} + \frac{16(1 - x_1)^{2}}{(2x_1 - x_1^{2})^{3}}&\\
        f''(1) &> 0 \implies min&\\
        x_2^{*} &= 2 - 1 = 1&
    \end{flalign*}

    Ответ: $f(1, 1) = 2$ --- экстремум минимума при заданном условии.

    Исходный код на языке \textit{Octave} для построения графиков представлен в листинге ниже:
    \insertcode{code.m}{octave}

    Визуализация, по которой писался код, приведена на рисунке~\ref{fig:1draft}.
    \imgh{1draft}{Черновик}

    \[
        f(x_1, x_2) = 25x_1x_2^{2},\ x_1 - 10x_2 = 1 
    .\] 

    Графическое описание примера приведено на рисунке~\ref{fig:2task}.
    \imgh{2task}{Пример 2}

    Решим задачу с помощью функции Лагранжа:
    \begin{flalign*}
        &L = f(x_1, x_2) + \lambda\phi(x_1, x_2) = 25x_1x_2^{2} + \lambda(x_1 - 10x_2 - 1),&
    \end{flalign*}
    где $\phi(x_1, x_2)$ --- уравнение связи, равное нулю, $\lambda$ --- множитель Лагранжа.

    Найдем частные производные ($\lambda = const$):
    \begin{flalign*}
        &L'_{x_1} = 25x_2^{2} + \lambda&\\
        &L'_{x_2} = 50x_1x_2 - 10\lambda&
        \intertext{Решим систему относительно $x_1$, $x_2$ и $\lambda$:}
        &\begin{cases}
            L'_{x_1} = 0\\
            L'_{x_2} = 0\\
            \phi(x_1, x_2) = 0
        \end{cases} \implies
        \begin{cases}
            25x_2^{2} + \lambda = 0\\
            5x_1x_2 - \lambda = 0\\
            x_1 - 10x_2 - 1 = 0
        \end{cases} \implies
        \begin{cases}
            \lambda = -25x_2^{2}\\
            5x_1x_2 + 25x_2^{2} = 0\\
            x_1 = 1 + 10x_2
        \end{cases}\\
        &5(1 + 10x_2)x_2 + 25x_2^{2} = 5x_2((1 + 10x_2) + 5x_2) = 0&\\
        &\begin{cases}
            x_2 = 0\\
            1 + 10x_2 + 5x_2 = 1 + 15x_2 = 0
        \end{cases} \implies
        \begin{cases}
            x_{21} = 0\\
            x_{22} = -\frac{1}{15}
        \end{cases}\\
        &x_{11} = 1&\\
        &x_{12} = 1 - \frac{10}{15} = \frac{1}{3}\\
        &\lambda_1 = 0&\\
        &\lambda_2 = \frac{-25}{225} = -\frac{1}{9}
    \end{flalign*}

    Для проверки достаточного условия, воспользуемся вторым дифференциалом функции Лагранжа:
    \begin{flalign*}
        &d^{2}L = L''_{x_1x_1}(dx_1)^{2} + 2L''_{x_1x_2}(dx_1dx_2) + L''_{x_2x_2}(dx_2)^{2}&\\
        &L''_{x_1x_1} = 0, L''_{x_1x_2} = 50x_2, L''_{x_2x_2} = 50x_1&\\
        &d^{2}L = 100x_2dx_1dx_2 + 50x_1(dx_2)^{2}&\\
        \intertext{Рассмотрим точку $(1, 0)$:}
        &d_1^{2}L = 50(dx_2)^{2} > 0&\\
        \intertext{Рассмотрим точку $(\frac{1}{3}, -\frac{1}{15})$:}
        &d_2^{2}L = -\frac{20}{3}dx_1dx_2 + \frac{50}{3}(dx_2)^{2}&
        \intertext{Так как дифференциалы знакопеременные, сравнение с нулем не очевидно --- выразим один из дифференциалов из уравнения связи:}
        &d(x_1 - 10x_2) = d(1)&\\
        &dx_1 - 10dx_2 = 0&\\
        &dx_1 = 10dx_2&\\
        &\frac{-20}{3}10(dx_2)^{2} + \frac{50}{3}(dx_2)^{2} = - \frac{150}{3}(dx_2)^{2} < 0&
    \end{flalign*}

    Ответ: $f(1, 0) = 0$ --- минимум, $f(\frac{1}{3}, -\frac{1}{15}) \approx 0,04$ --- максимум. 

    \[
        f(x_1, x_2) = 2x_1 + 3x_2,\ x_1^{2} + x_2^{2} = 1
    .\]

    Графическое описание примера приведено на рисунке~\ref{fig:3task}.
    \imgh{3task}{Пример 3}

    \begin{flalign*}
        &L = 2x_1 + 3x_2 + \lambda(x_1^{2} + x_2^{2} - 1)&\\
        &L'_{x_1} = 2 + 2\lambda x_1&\\
        &L'_{x_2} = 3 + 2\lambda x_2&\\
        &\begin{cases}
            2 + 2\lambda x_1 = 0\\
            3 + 2\lambda x_2 = 0\\
            x_1^{2} + x_2^{2} - 1 = 0
        \end{cases} \implies
        \begin{cases}
            x_1 = - \frac{2}{2\lambda} = - \frac{1}{\lambda}\\
            x_2 = - \frac{3}{2\lambda}\\
            \frac{1}{\lambda^{2}} + \frac{9}{4\lambda^{2}} - 1 = 0
        \end{cases}\\
        &\frac{4 + 9 - 4\lambda^{2}}{4\lambda^{2}} = 0&\\
        &\lambda^{2} = \frac{13}{4} \implies \lambda_{1,2} = \pm \frac{\sqrt{13}}{2}&\\
        &x_{11} = - \frac{2}{\sqrt{13}}&\\
        &x_{12} = \frac{2}{\sqrt{13}}&\\
        &x_{21} = - \frac{3 \cdot 2}{2 \cdot \sqrt{13}} = - \frac{3}{\sqrt{13}}&\\
        &x_{22} = \frac{3}{2} \cdot \frac{2}{\sqrt{13}} = \frac{3}{\sqrt{13}}&
        \intertext{Рассмотрим достаточное условие существования экстремума в матичной форме в найденных точках:}
        &A = 
        \begin{bmatrix}
            0 & \phi'_{x_1} & \phi'_{x_2}\\
            \phi'_{x_1} & L''_{x_1x_1} & L''_{x_1x_2}\\
            \phi'_{x_2} & L''_{x_2x_1} & L''_{x_2x_2}
        \end{bmatrix} = 
        \begin{bmatrix}
            0 & 2x_1 & 2x_2\\
            2x_1 & 2\lambda & 0\\
            2x_2 & 0 & 2\lambda
        \end{bmatrix}\\
        &A_1 =
        \begin{bmatrix}
            0 & -\frac{4}{\sqrt{13}} & -\frac{6}{\sqrt{13}}\\
            -\frac{4}{\sqrt{13}} & \sqrt{13} & 0\\
            -\frac{6}{\sqrt{13}} & 0 & \sqrt{13}
        \end{bmatrix}\\
        &\Delta_1 = \sum\limits_{j=1}^{3}(-1)^{1+j}M_{1j}a_{1j} = \frac{4}{\sqrt{13}}(-4) - \frac{6}{\sqrt{13}}(6) < 0&\\
        &A_2 =
        \begin{bmatrix}
            0 & \frac{4}{\sqrt{13}} & \frac{6}{\sqrt{13}}\\
            \frac{4}{\sqrt{13}} & -\sqrt{13} & 0\\
            \frac{6}{\sqrt{13}} & 0 & -\sqrt{13}
        \end{bmatrix}\\
        &\Delta_2 = -\frac{4}{\sqrt{13}}(-4) + \frac{6}{\sqrt{13}}(6) > 0&
    \end{flalign*}
    Ответ: $f(-\frac{2}{\sqrt{13}}, -\frac{3}{\sqrt{13}}) \approx -3,6$ --- минимум, $f(\frac{2}{\sqrt{13}}, \frac{3}{\sqrt{13}}) \approx 3,6$ --- максимум.

    \[
        y = f(x_1, x_2) = x_1^2 + x_2^2 - x_1x_2 - x_1 - x_2,\ X: x_1 \geq 0, x_2 \geq 0, x_1 + x_2 \leq 3
    .\] 

    Графическое описание примера приведено на рисунке~\ref{fig:4task}.
    \imgh{4task}{Пример 4}

    \begin{tikzpicture}
        \begin{axis}[
            axis x line=center,
            axis y line=center,
            xmin=-1, xmax=4,
            xlabel={$x_1$},
            ylabel={$x_2$},
            samples at={-1,-0.9,...,4}
        ]
            \addplot[red, smooth]{3 - x};
            \draw[red] (0,0) -- (0,4);
            \draw[red] (0,0) -- (4,0);

            \fill[red,opacity=0.2] (0,0) -- (0,3) -- (3,0);

            \filldraw (1,1) circle (2pt) node[above left] {$M_1$};
            \filldraw (0,0.5) circle (2pt) node[above left] {$M_2$};
            \filldraw (0,0) circle (2pt) node[above left] {$M_3$};
            \filldraw (0,3) circle (2pt) node[above right] {$M_4$};
            \filldraw (0.5,0) circle (2pt) node[above] {$M_5$};
            \filldraw (3,0) circle (2pt) node[above] {$M_6$};
            \filldraw (1.5,1.5) circle (2pt) node[above right] {$M_7$};
        \end{axis}
    \end{tikzpicture}

    Найдем стационарные точки $M$ внутри $X$:
    \begin{flalign*}
        &y'_{x_1} = 2x_1 - x_2 - 1&\\
        &y'_{x_2} = 2x_2 - x_1 - 1&\\
        &\begin{cases}
            2x_1 - x_2 - 1 = 0\\
            2x_2 - x_1 - 1 = 0
        \end{cases} \implies
        \begin{cases}
            x_2 = 2x_1 - 1\\
            4x_1 - 2 - x_1 - 1 = 0
        \end{cases} \implies
        \begin{cases}
            x_2 = 1\\
            x_1 = 1
        \end{cases}\\
        &M_1(1, 1),\ \mathbf{f(M_1) = -1}&
    \end{flalign*}

    Исследуем границу области:
    \begin{flalign*}
        &\begin{cases}
            x_1 = 0\\
            y = x_1^2 + x_2^2 - x_1x_2 - x_1 - x_2
        \end{cases} \implies y = x_2^2 - x_2\\
        &y' = 2x_2 - 1 = 0 \implies x_2 = \frac{1}{2}&\\
        &M_2(0, \frac{1}{2}),\ \mathbf{f(M_2) = \frac{1}{4} - \frac{1}{2} = - \frac{1}{4}}&\\
        &M_3(0, 0),\ \mathbf{f(M_3) = 0}&\\
        &M_4(0, 3),\ \mathbf{f(M_4) = 6}&\\
        \\
        &\begin{cases}
            x_2 = 0\\
            y = x_1^2 + x_2^2 - x_1x_2 -x_1 - x_2
        \end{cases} \implies y = x_1^2 - x_2\\
        &f' = 2x_1 - 1 = 0 \implies x_1 = \frac{1}{2}&\\
        &M_5(\frac{1}{2}, 0),\ \mathbf{f(M_5) = - \frac{1}{4}}&\\
        &M_6(3, 0),\ \mathbf{f(M_6) = - 6}&\\
        \\
        &\begin{cases}
            x_2 = 3 - x_1\\
            y = x_1^2 + x_2^2 - x_1x_2 - x_1 - x_2
        \end{cases} \implies y = x_1^2 + (3 - x_1)^2 - x_1(3 - x_1) - x_1 - (3 - x_1) =\\
        &= x_1^2 + 9 - 6x_1 + x_1^2 - 3x_1 + x_1^2 - x_1 - 3 + x_1 = 3x_1^2 - 9x_1 + 6&\\
        &f' = 6x_1 - 9 = 0 \implies x_1 = \frac{9}{6} = \frac{3}{2},\ x_2 = 1,5&\\
        &M_7(\frac{3}{2}, \frac{3}{2}),\ \mathbf{f(M_7) = \frac{9}{4} + \frac{9}{4} - \frac{9}{4} - \frac{3}{2} - \frac{3}{2} = \frac{9}{4} - \frac{6}{2} = - \frac{3}{4}}&
    \end{flalign*}

    Ответ: $\underset{X}{max}y = f(3; 0) \cup f(0; 3) = 6$, $\underset{X}{min}y = f(1; 1) = -1$.

\subsection*{Вывод}

Изучил аналитические методы нахожения условного экстремума функции двух переменных, написал
программную реализацию вывода графиков и их сечений.

\end{document}
